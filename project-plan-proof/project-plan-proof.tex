\documentclass[sigconf]{acmart}
\usepackage{graphicx}
\usepackage{booktabs} % For formal tables
\usepackage{enumitem}
\makeatletter
\def\subsubsection{\@startsection{subsubsection}{3}{10pt}%
                                 {-.5\baselineskip \@plus -2\p@ \@minus -.2\p@}%
                 {3.5\p@}{\subsubsectionfont}}
\makeatother

% Copyright
\setcopyright{none}

% DOI
\acmDOI{10.475/123_4}

% ISBN
\acmISBN{123-4567-24-567/08/06}

\acmConference[CPSC 311]{Intro. PL}{2018W1}{UBC Vancouver}
\acmYear{2018}
\copyrightyear{2018}

\acmArticle{4}
\acmPrice{15.00}

\begin{document}
\title{Project Plan/Proof-of-Concept:\\ Refactoring an Open-source System with AspectJ}

\author{Tony Kong}
\affiliation{%
  \institution{University of British Columbia}
  \city{Vancouver}
  \state{BC}
}
\email{y2k0b@ugrad.cs.ubc.ca}

\author{Raymond Situ}
\affiliation{%
  \institution{University of British Columbia}
  \city{Vancouver}
  \state{BC}
}
\email{r7p0b@ugrad.cs.ubc.ca}

\author{Kevin Wong}
\affiliation{%
  \institution{University of British Columbia}
  \city{Vancouver}
  \state{BC}
}
\email{b2j0b@ugrad.cs.ubc.ca}

\author{Benjamin Hwang}
\affiliation{%
  \institution{University of British Columbia}
  \city{Vancouver}
  \state{BC}
}
\email{r6o0b@ugrad.cs.ubc.ca}

\author{James Yoo}
\affiliation{%
  \institution{University of British Columbia}
  \city{Vancouver}
  \state{BC}
}
\email{l4k0b@ugrad.cs.ubc.ca}

\begin{abstract}
AspectJ is a powerful extension to the Java language which enables developers to apply a separation of concerns to a system with a high degree of crosscutting and scattering. This report details our refactoring of a small open-source project \footnote{https://github.com/jyoo980/aop-spaceinvaders} using AspectJ, and describes a possible approach to refactoring a larger-scale software system with the lessons learned and insights provided from our initial work.
\end{abstract}

\begin{CCSXML}
<ccs2012>
<concept>
<concept_id>10011007.10011006.10011008.10011024.10011038</concept_id>
<concept_desc>Software and its engineering~Frameworks</concept_desc>
<concept_significance>300</concept_significance>
</concept>
</ccs2012>
\end{CCSXML}
\ccsdesc[300]{Software and its engineering~Frameworks}

\keywords{Software evolution, refactoring, aspect-oriented programming}
\maketitle

\end{document}

