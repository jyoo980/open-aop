%%%% Proceedings format for most of ACM conferences (with the exceptions listed below) and all ICPS volumes.
\documentclass[sigconf]{acmart}

\usepackage{booktabs} % For formal tables

 \setcopyright{none}

\acmConference[CPSC 311]{Intro. PL}{2018W1}{UBC Vancouver}
\acmYear{2018}
\copyrightyear{2018}

\acmArticle{4}


\begin{document}
\title{Proposal \protect\\ Open AOP: Analysing Cross-Cutting Concerns \protect\\ in Open Source Software}

\author{Tony Kong}
\affiliation{%
  \institution{University of British Columbia}
  \city{Vancouver}
  \state{BC}
}
\email{y2k0b@ugrad.cs.ubc.ca}

\author{Raymond Situ}
\affiliation{%
  \institution{University of British Columbia}
  \city{Vancouver}
  \state{BC}
}
\email{r7p0b@ugrad.cs.ubc.ca}

\author{Kevin Wong}
\affiliation{%
  \institution{University of British Columbia}
  \city{Vancouver}
  \state{BC}
}
\email{b2j0b@ugrad.cs.ubc.ca}

\author{Benjamin Hwang}
\affiliation{%
  \institution{University of British Columbia}
  \city{Vancouver}
  \state{BC}
}
\email{r6o0b@ugrad.cs.ubc.ca}

\author{James Yoo}
\affiliation{%
  \institution{University of British Columbia}
  \city{Vancouver}
  \state{BC}
}
\email{l4k0b@ugrad.cs.ubc.ca}

\begin{abstract}
Aspect-oriented Programming is a technique for improving the modularity of a system by via a construct called an Aspect, which encapsulates the cross-cutting concerns and enforces a separation of concerns between business and application logic. In \textit{Open AOP}, we analyze a number of open-source Java software systems, measure the extent of cross-cutting extant prior to refactoring using AspectJ. We then analyze the evolved system, paying particular attention to cross-cutting, coupling, and cohesion.  
\end{abstract}

\keywords{Aspect-oriented Programming, Cross-cutting concern, Joinpoint, Pointcut, Advice, Open-source}
\maketitle

\end{document}

