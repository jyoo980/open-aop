\documentclass[sigconf]{acmart}

\usepackage{booktabs} % For formal tables


% Copyright
\setcopyright{none}


% DOI
\acmDOI{10.475/123_4}

% ISBN
\acmISBN{123-4567-24-567/08/06}

\acmConference[CPSC 311]{Intro. PL}{2018W1}{UBC Vancouver}
\acmYear{2018}
\copyrightyear{2018}

\acmArticle{4}
\acmPrice{15.00}

\begin{document}
\title{AspectJ: a Pragmatic Approach to Cross-cutting Concerns}

\author{Tony Kong}
\affiliation{%
  \institution{University of British Columbia}
  \city{Vancouver}
  \state{BC}
}
\email{y2k0b@ugrad.cs.ubc.ca}

\author{Raymond Situ}
\affiliation{%
  \institution{University of British Columbia}
  \city{Vancouver}
  \state{BC}
}
\email{r7p0b@ugrad.cs.ubc.ca}

\author{Kevin Wong}
\affiliation{%
  \institution{University of British Columbia}
  \city{Vancouver}
  \state{BC}
}
\email{b2j0b@ugrad.cs.ubc.ca}

\author{Benjamin Hwang}
\affiliation{%
  \institution{University of British Columbia}
  \city{Vancouver}
  \state{BC}
}
\email{r6o0b@ugrad.cs.ubc.ca}

\author{James Yoo}
\affiliation{%
  \institution{University of British Columbia}
  \city{Vancouver}
  \state{BC}
}
\email{l4k0b@ugrad.cs.ubc.ca}

\begin{abstract}
This paper details AspectJ, a general-purpose Aspect-oriented programming language built on top of Java. We explore AspectJ's mechanisms that enable software developers to encapsulate cross-cutting concerns. We also detail how the language allows developers to better reason about their systems by enforcing a separation of concerns between business logic and cross-cutting logic.

\end{abstract}
\begin{CCSXML}
<ccs2012>
<concept>
<concept_id>10011007.10011006.10011008.10011024.10011038</concept_id>
<concept_desc>Software and its engineering~Frameworks</concept_desc>
<concept_significance>300</concept_significance>
</concept>
</ccs2012>
\end{CCSXML}
\ccsdesc[300]{Software and its engineering~Frameworks}

\keywords{AspectJ, Aspect-oriented Programming, Pointcut, Joinpoint, Advice, Cross-cutting concern, application logic, business logic}


\maketitle

\end{document}

